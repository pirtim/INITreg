\section*{Koło, jego cele i sposoby ich realizacji~~}
\section{}
\begin{enumerate}
\item Studenckie Koło Innowacji i Transferu Technologii ``INIT'', zwane dalej Kołem, jest Uczelnianą Organizacją Studencką, działającą na podstawie ustawy --  Prawo o szkolnictwie wyższym  (Dz.~U.~Nr 164, poz. 1365, z późn. zm.), Statutu Uniwersytetu Warszawskiego oraz niniejszego Regulaminu.

\item Siedzibą Koła jest Uniwersytecki Ośrodek Transferu Technologii Uniwersytetu Warszawskiego.
\end{enumerate}

\section{}
Celami Koła są:
\begin{enumerate}[label=\alph*)]
\item popularyzacja tematyki komercjalizacji wśród studentów,
\item upowszechnianie wiedzy o własności intelektualnej,
\item promocja nowych technologii i innych rozwiązań aplikacyjnych,
\item zachęcanie studentów do opracowywania ambitnych rozwiązań praktycznych, w szczególności o potencjale komercjalizacyjnym,
\item integracja istniejących Uczelnianych Organizacji Studenckich,
\item integracja środowisk naukowych z podmiotami biznesowymi.
\end{enumerate}
\section{}

Koło realizuje swoje cele poprzez:
\begin{enumerate}[label=\alph*)]
\item promocję lub organizację konkursów, projektów, wystaw i wydarzeń,
\item upowszechnianie wiedzy o działalności Uniwersyteckiego Ośrodka Transferu Technologii,
\item rozwijanie współpracy z innymi podmiotami zajmującymi się innowacjami i komercjalizacją,
\item wykonywanie prac przedwdrożeniowych lub prototypowanie,
\item prowadzenie, publikacja oraz promocja badań naukowych lub aplikacyjnych,
\item podejmowanie innych działań, dążących do osiągnięcia celów Koła.
\end{enumerate}


\section*{Członkostwo~~~}
\section{}
Członkiem Koła może być każdy student i doktorant Uniwersytetu Warszawskiego.

\section{}
\begin{enumerate}
\item Nabycie członkostwa Koła następuje wskutek uchwały Zarządu Koła akceptującej pisemną deklarację członkowską.
\item Utrata członkowska następuje na skutek: uchwały Zarządu Koła; oświadczenia woli wystąpienia z Koła, podanego do wiadomości Zarządu; zgonu członka; skreślenia z listy studentów Uniwersytetu Warszawskiego.
\item Od decyzji Zarządu o wykluczeniu z Koła członek może się odwołać do Zebrania Członków.
\end{enumerate}

\section{}
Członek Koła ma prawo do:
\begin{enumerate}[label=\alph*)]
\item czynnego i biernego prawa wyborczego do władz Koła,
\item udziału w pracach Koła, korzystania ze wszystkich urządzeń, obiektów i stwarzanych przez Koło możliwości,
\item używania odznak i znaków Koła, a także jego reprezentowania na wszelkich imprezach i spotkaniach o charakterze nie kolidującym z celami i filozofią działania Koła,
\item zgłaszania opinii, wniosków i postulatów pod adresem władz Koła,
\item uczestniczenia w Zebraniu Członków.
\end{enumerate}

\section{}
Do obowiązków członków należy:
\begin{enumerate}[label=\alph*)]
\item aktywne uczestniczenie w pracach i realizacji celów Koła,
\item stosowanie się w swojej działalności do uchwał oraz wytycznych władz Koła,
\item regularne opłacanie składek członkowskich na rzecz Koła.
\end{enumerate}

\section*{Władze Koła~~~}
\section{}
Do władz Koła należy:
\begin{enumerate}[label=\alph*)]
\item Zebranie Członków,
\item Zarząd.
\end{enumerate}

\section{}
\begin{enumerate}
\item Władze Koła podejmują decyzje zwykłą większością głosów, z wyłączeniem \S \ref{kworum} ust. 2.
\item Głosowania władz Koła są jawne dla członków Koła. Zebranie Członków może zadecydować o głosowaniu w sposób tajny.
\item W wypadku impasu decydujący jest głos Prezesa. Zasada ta nie ma zastosowania w przypadku głosowania w sposób tajny oraz nad wyborem członków Zarządu.
\end{enumerate}

\section*{Zebranie Członków~~}
\section{}
Do kompetencji Zebrania Członków, stanowiącego najwyższą władzę w Kole, należy:
  \begin{enumerate}[label=\alph*)]
  \item zmiana nazwy Koła,
  \item uchwalanie i zmiany Regulaminu Koła,
  \item odwołanie i wybór członków Zarządu, 
  \item ustalenie liczby członków Zarządu,
  \item ustalanie wysokości składek członkowskich,
  \item rozpatrywanie sprawozdań z działalności Zarządu,
  \item rozpatrywanie odwołań od uchwał Zarządu o wykluczeniu członka z Koła,
  \item odwołanie Opiekuna Naukowego,
  \item podjęcie decyzji o rozwiązaniu Koła,
  \item rozpatrywanie innych spraw wniesionych przez członków Koła.
  \end{enumerate}

\section{}
\label{Tryb}
\begin{enumerate}
\item Zebranie Członków zwoływane jest przez Prezesa: przynajmniej raz w roku, w terminie 45 dni od rozpoczęcia roku akademickiego lub na wniosek co najmniej 35\% członków Koła lub na wniosek Zarządu.
\item Zwołujący Zebranie Członków ma obowiązek powiadomić o jego terminie, miejscu i porządku obrad, drogą elektroniczną za pomocą środków komunikowania się na odległość, członków Koła oraz Opiekuna Naukowego co najmniej 14 dni przed wyznaczonym terminem.
\item Zebranie Członków może odbyć się bez wcześniejszego zwołania, w trybie ad hoc.
\item Na głosowaniach podczas Zebrania Członków w trybie ad hoc do oddanych głosów dolicza się głosy przeciw w liczbie odpowiadającej liczbie nieobecnych Członków Koła.
\end{enumerate}

\section{}
\begin{enumerate}
\label{kworum}
\item Kworum potrzebne do podjęcia uchwał na Zebraniu Członków wynosi, co najmniej połowę członków Koła.
\item Zebranie Członków wybiera członków Zarządu poprzez głosowanie w ordynacji preferencyjnej metodą głosu alternatywnego (zgodnie z procedurą opisaną w: Rzążewski, Kazimierz, Wojciech Słomczyński, Karol Życzkowski, i Marek Wójcikiewicz. \textit{¡Każdy Głos Się Liczy!: Wędrówka Przez Krainę Wyborów}. Strony: 171-172).
\item Do podjęcia decyzji o rozwiązaniu koła wymagana jest większość kwalifikowana $^2/_3$ głosów.
\end{enumerate}

\section{}
Jeżeli kworum, o którym mowa w \S \ref{kworum} ust. 1, nie zostanie osiągnięte, Prezes zwołuje Zebranie Członków, z zachowaniem trybu określonego w \S \ref{Tryb}, na którym wymóg kworum nie obowiązuje, chyba że Regulamin dla ważności poszczególnych uchwał wymaga kwalifikowanej większości głosów. 

\section{}
Wniosek o zmianę Regulaminu, wraz z proponowanymi zmianami, składa się na ręce Zarządu najpóźniej 4 dni przed Zgromadzeniem Członków.

\section*{Zarząd~~~}
\section{}
W kompetencjach Zarządu leżą wszystkie decyzje niezastrzeżone do kompetencji Zebrania Członków, a w szczególności:
\begin{enumerate}[label=\alph*)]
\item dysponowanie majątkiem Koła,
\item wybór Opiekuna Naukowego,
\item określenie identyfikacji graficznej Koła.
\end{enumerate}

\section{}
\begin{enumerate}
\item Zarząd składa się z Prezesa, Wiceprezesa ds. finansowych zwanego dalej Skarbnikiem oraz innych członków jeżeli Zebranie Członków tak postanowi.
\item Kadencja Zarządu trwa jeden rok.
\item W wypadku zaistnienia wakatu w składzie Zarządu, Zarząd może dokonać uzupełnienia składu drogą
kooptacji.
\end{enumerate}

\section{}
\begin{enumerate}
\item Prezes kieruje pracami Zarządu oraz reprezentuje Koło na zewnątrz.
\item Prezes w terminie do dnia 31 stycznia każdego roku składa Rektorowi coroczne sprawozdanie z działalności Koła za rok poprzedni. Sprawozdanie to jest jawne dla członków Koła oraz Opiekuna Naukowego.
\item Prezes zobowiązany jest przedkładać Rektorowi wszelkie zmiany w Regulaminie Koła oraz zmiany członków Zarządu.
\item Prezes w wymaganych terminach składa odpowiednie rozliczenia z przyznanych środków materialnych do odpowiednich organów uczelni.
\item Prezes może delegować określony zakres swoich kompetencji innemu członkowi Zarządu. Delegacja może zostać odwołana w każdym czasie.
\end{enumerate}

\section*{Majątek Koła~~~}
\section{}
Majątek Koła powstaje z: składek członkowskich, darowizn, dotacji, dochodów nadzwyczajnych.
 
\section{}
Składki członkowskie pobiera Skarbnik.
 
\section{}
Skarbnik jest zobowiązany do przygotowania sprawozdania finansowego do 31 stycznia każdego roku. Sprawozdanie finansowe podlega zatwierdzeniu przez Zarząd oraz jest jawne dla członków Koła oraz Opiekuna Naukowego.

\section{}
W przypadku przyznania przez organ uczelni środków materialnych Kołu, Skarbnik zobowiązany jest do przygotowania rozliczenia z przyznanych środków co najmniej raz w semestrze, w terminie ustalonym przez Radę Konsultacyjną ds. Studenckiego Ruchu Naukowego.

\section{}
Zobowiązania zaciągają w imieniu Koła Prezes i Skarbnik działający łącznie. Inne wiążące akty podpisuje w imieniu Koła Prezes i inny członek Zarządu, względnie, upoważniony przez Zarząd, samodzielnie jeden członek Zarządu.
 
\section{}
W nadzwyczajnych przypadkach Prezesowi przysługuje prawo do samodzielnego zaciągnięcia zobowiązania, z zastrzeżeniem jednak potwierdzenia jego działania decyzją Zarządu w ciągu 7 dni od daty zaciągnięcia zobowiązania.
 
\section{}
W przypadku likwidacji Koła jego majątek zostaje przeniesiony na rzecz Uniwersytetu Warszawskiego.